
# GANS

Generative Adversarial Networks (GANs) are a class of deep learning models introduced by Ian Goodfellow and his colleagues in 2014, designed to generate realistic data samples. GANs consist of two neural networks, a generator and a discriminator, that are trained simultaneously in a competitive setting. The generator creates fake data samples, while the discriminator evaluates whether the samples are real (from the training data) or fake (generated by the generator). The generator's objective is to produce data that can deceive the discriminator, while the discriminator aims to correctly distinguish between real and fake data. This adversarial process continues until the generator produces data that are indistinguishable from real data. GANs have been successfully applied in various fields, such as image generation, style transfer, and even creating deepfake videos.


# Aoutoencoders
Autoencoders are a type of artificial neural network designed for unsupervised learning, primarily used for tasks like dimensionality reduction, feature learning, and data reconstruction. They consist of two main components: an encoder and a decoder. The encoder compresses the input data into a lower-dimensional representation, known as the latent space, capturing the most essential features. The decoder then reconstructs the original data from the th


Variational AutoEncoders
# AI

A Variational Autoencoder (VAE) is a generative model that combines principles from deep learning and probabilistic modeling. It consists of an encoder that compresses input data into a latent space representation, followed by a decoder that reconstructs the data from this compressed form. Unlike traditional autoencoders, VAEs introduce a probabilistic element by mapping inputs to a distribution rather than a fixed point in latent space, typically using a Gaussian distribution. During training, the VAE optimizes both the reconstruction error and the Kullback-Leibler divergence, ensuring that the latent space conforms to the desired distribution. This makes VAEs powerful tools for generating new data similar to the training set, and they are widely used in tasks such as image generation, anomaly detection, and data compression.


Diffusion Models



# Diffusion models
Diffusion models are a class of generative models that create data by reversing a diffusion process, which involves gradually transforming a structured signal into noise and then learning how to reverse this process to reconstruct the original data. The forward process systematically adds noise to the data over a series of steps, eventually producing a distribution close to pure noise. The model then learns to reverse this noising process, generating realistic samples by iteratively denoising the data. Diffusion models have gained popularity because of their ability to generate high-quality and diverse outputs, and they have been applied in various fields, including image generation, text-to-image synthesis, and even molecular design. Their robustness and flexibility have made them a strong competitor to other generative models such as GANs.

# Stable diffusion
Stable Diffusion is a generative model designed for creating high-quality images through a diffusion process, which gradually transforms random noise into a coherent image. Unlike traditional generative models, Stable Diffusion operates by learning the reverse of a diffusion process, starting with a noisy image and iteratively refining it to produce a clear and realistic output. This model leverages deep learning techniques, particularly denoising autoencoders, to estimate the noise in an image and subtract it, thereby generating increasingly detailed images with each iteration. Stable Diffusion is particularly noted for its ability to generate complex, high-resolution images and has been applied in various domains, including art creation, style transfer, and synthetic data generation. Its stability and quality make it a powerful tool in the field of image synthesis.

# Noise Schedulers

Noise schedulers are integral components of diffusion models, responsible for controlling the amount and distribution of noise added to data during the forward diffusion process. These schedulers determine how noise is introduced at each step, typically following a predefined schedule that governs the variance or scale of the noise across time steps. The choice of noise schedule influences the stability and quality of the generated data, as it impacts the model's ability to learn the reverse process effectively. Common noise schedules include linear, cosine, and exponential schemes, each with its trade-offs in terms of sample diversity and convergence speed. By carefully designing noise schedulers, researchers can enhance the performance of diffusion models, leading to more accurate and high-fidelity data generation.

#
U-Net is a convolutional neural network architecture originally developed for biomedical image segmentation, introduced by Olaf Ronneberger and his colleagues in 2015. The architecture is characterized by its U-shaped structure, consisting of a contracting path (encoder) and an expansive path (decoder). The encoder captures context by progressively downsampling the input image through convolutional and pooling layers, while the decoder reconstructs the image at full resolution.




\begin{enumerate}
    \item VAE
    
\end{enumerate}