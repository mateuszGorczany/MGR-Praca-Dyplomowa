\subsection{Programming setup}
Data augmentation using deep learning is a highly complicated process, so choosing the right tooling plays a crucial role. Lack of knowledge about modern technologies that help with programming environment setup may lead to many hours of additional time spent on the project that are not specifically related to the core analysis. In order to reduce the time spent on configuration errors, modern state-of-the-art technologies were employed to mitigate this issue.  

\paragraph{Git}\mbox{}\\
\indent Git\footnote{\url{https://cookiecutter-data-science.drivendata.org/}} is a software code versioning system created in 2005. It allows the user to have a local copy of the project code and propose changes to the main copy stored on the remote server, such as Github. Thanks to this tool, collaboration between different programmers is possible. Additionally, unsuccessful attempts or flaws in the implementation may be reverted to the original state.
\paragraph{Github}\mbox{}\\
\indent Enables the user to store the git repository on the remote server. 
This project's code: model implementations, environment requirements, experiment configurations, generated artifacts, and more are stored in the Github repository "ct-images-augmentation" \footnote{\url{https://github.com/mateuszGorczany/ct-images-augmentation/}}.
\paragraph{Data Science project structure}\mbox{}\\
\indent Data Science projects like this often take different forms. Lack of standarization - where and what is stored, may lead to decrease of programmers' productivity over time. Data Science cookie-cutter authors came in the front of this issue. They created a standardized Data Science project structure\footnote{\url{https://cookiecutter-data-science.drivendata.org/}} that can be used to implement a variety of Machine Learning models. 
The mentioned project structure was adjusted to the needs of this project and had taken the following form:

\lstinputlisting[basicstyle=\ttfamily\tiny]{detailed_engineering/folder_structure.txt}


The existing project was inspired by the template.
\paragraph{VSCode Remote SSH}\mbox{}\\
\indent Working on a remote computer can be a difficult task to handle. Remote environments often lack the tools that make a programmer effective. Especially a code editor and extensions that a programmer is used to work with and which make a programmer effective in writing code and detecting bugs. VSCode Remote SSH\footnote{\url{https://marketplace.visualstudio.com/items?itemName=ms-vscode-remote.remote-ssh}} enables a programmer that uses Visual Studio Code to connect to a remote machine with the current editor, so it is almost indistinguishable from the experience of working on a local machine. This way no time was spent on remote code editor setup and extensions installation and configuration. 

\paragraph{Devbox}\mbox{}\\
\indent Remote computers often have their administrators. Almost always they are only ones who are in the super-users' group called "sudoers" in a Linux environment. It helps them keep the machine secure, but also makes them only persons who are able to install additional software, which is later available for everyone. On the other hand, it raises an issue of additional package installation, like specific Python versions and CLI's (Command Line Interfaces) that are required by a project or make a programmer productive.  
To mitigate this issue, a programmer can install software only for himself. However, with the standard Debian/Ubuntu package installer, "apt", it is hard to employ this strategy since many commands still require sudo access. 

Devbox\footnote{\url{https://www.jetify.com/devbox}} makes it possible to create an environment per project and specify the exact versions of software that should be installed. It can be integrated with Git and Github to make it possible for other programmers to reproduce exactly the same environment on their computers. 

Devbox uses the "Nix" package installer under the hood. It is currently the biggest package manager that exists in the Unix landscape. Nix allows us to install most of the open source software that has ever been created. Unfortunatelly, it uses a functional language that has been specifically created for it. Devbox make it possible to define programs that should be installed in a "json" file with a specific structure and pin a specific version of the software. Then it creates a lock file in which exact versions for each platform with hashes of the programs are stored. It makes it possible to fully recreate the programming environment created on one computer on another.
 

\paragraph{Direnv}\mbox{}\\
\indent When it comes to working with programming projects, the programmer has to switch between them. It requires a Python programmer to change the Python environment to the right one that consists of everything that the project requires to run. Direnv\footnote{\url{https://direnv.net/}} automatically activates the right environment when a directory of the project is entered, so the programmer does not have to do it manually and remember about it.

In the project Devbox environment and Python environment are automatically enabled when a programmer enters the project directory (and has direnv installed).
\paragraph{Python}\mbox{}\\
\indent Numerous programming languages such as Matlab, Julia, and Elixir support the development of deep learning projects. However, Python stands out as the primary language for most research and advancements in Artificial Intelligence. Therefore, it was the natural choice for this project.
\paragraph{Rye}\mbox{}\\
\indent Almost every Python project uses external libraries written by other programmers. It makes the programmer not reinvent the wheel again and dramatically reduces the project implementation time. However, it raises the issue of external package version management. To tackle this issue, Python programmers came up with the 'requirements.txt' file that should include all external, nonstandard, dependencies that the project requires in order to make the environment reproducible on the other computer. In addition, they introduced the `pip` package manager that allows them to be installed. 
Rye\footnote{\url{https://rye.astral.sh/}} is a more modern solution to this problem. It enables the user to create a Python package from the existing project, define external dependencies with their exact versions, and create a Python virtual environment. 

What is more - it makes it possible to use a modern replacement for the `pip` package manager - uv. UV\footnote{\url{https://github.com/astral-sh/uv}} is the modern package installer written in Rust that is in some cases even 100 times faster than the predecessor.  
\paragraph{Pytorch}\mbox{}\\
\indent Pytorch\cite{NEURIPS2019_9015} is a Deep Learning framework for Python programmers. It makes it possible to create neural networks and train them on a GPU. Pytorch is currently the most widely used framework in deep learning.
Most of the nowadays research on the neural networks is done using this framework, so it was an obvious choice, since the project was meant to use state of the art neural network written by other programmers. Using this framework made it possible to reuse their code without reimplementation. 
\paragraph{Pytorch Lightning}\mbox{}\\
\indent Implementing neural networks in Pytorch always consists of some repetitive steps, such as recording, training callbacks, retraining and using multiple GPUs. Pytorch Lightning\cite{Falcon_PyTorch_Lightning_2019} authors noticed this issue and created a standard template for the training of deep learning models that proposes a standard structure for model training, making code more readable and easier to modify. Additionally, they employ researcher with many tools that make the programmer more effective, like callbacks mechanism, floating point precision switching and most importantly, multiple GPU training. With Pytorch Lightning multiple GPU training is possible with only one parameter switch. 

\paragraph{Monai}\mbox{}\\
\indent Monai\cite{Cardoso_MONAI_An_open-source_2022} is a collection of frameworks aimed at helping research and clinical collaboration in Medical Imaging. The visualization functions of the framework as well as implementations of deep learning architectures were used in the project.
\paragraph{Nohup}\mbox{}\\
\indent CLI that keeps the process running even after the user's log-out/session ends. It was used in the project's Makefile\footnote{\url{https://github.com/mateuszGorczany/ct-images-augmentation/blob/main/Makefile}} 
\newpage
\subsection{Hardware}
Training neural networks requires substantial computational resources, particularly for the latest architecture models. These models typically consume significant memory, making it impractical to train them on personal computers. To address this, two Nvidia Quadro RTX 8000 GPUs, each equipped with 48GB of RAM, were utilized. The computer was located at the WFiIS Faculty at the AGH UST in Kraków. 

\subsection{Hardware utilisation - Multiple GPU utilisation techniques}
Training on a single GPU is straightforward. However, when a single card falls short, the challenge of leveraging multiple GPUs arises. The solution varies according to the underlying reason for employing multiple GPUs.
\paragraph{Distributed data parallel}\mbox{} \\
\indent In case a single GPU can handle training a neural network, but someone wanted to speed up the process with an additional GPU, the Distributed Data Parallel (DDP) technique can help. This approach distributes copies of a neural network model and data across all GPUs, allowing for concurrent training of multiple model instances. However, during the backpropagation phase, the gradients are synchronized and averaged.

This technique was used to train most of the models presented later, since it sped up the training twice. 

\paragraph{Fully Sharded Data Parallel}\mbox{} \\
\indent When a model cannot fit in one GPU memory, then the fully sharded data parallel (FSDP) technique can be used to solve the problem. This technique splits the model layers between cards and then uses cross-device communication techniques to train the model. However, this solution slows down the training due to the overhead of cross-device communication.

\subsection{Training monitoring}\mbox{}\\
\indent Training of neural networks often takes a lot of time - hours, days, weeks and sometime even months. It is crucial to monitor it in order not to waste resources on unsuccessful training. Whats more - when there are many models that are trained or there are many attempts of training it is easy to loose track of what has been accomplished, which models were already trained and what were the results. 
Monitoring enables the researcher to have visibility of important metrics for the model. Thus, unsuccessful training may be stopped ahead of time. It also helps to visualize training progress, model output and makes it possible to share the results with other researchers during the training. 
\paragraph{Tensorboard}\mbox{}\\
\indent Tensorboard\footnote{\url{https://www.tensorflow.org/tensorboard}} is an open source project that enables user to locally monitor training of a neural networks. It makes it possible to:
\begin{itemize}
    \item gather metrics,
    \item visualize model outputs,
    \item organize model trainings in the friendly user interface.
\end{itemize}


\paragraph{Wandb}\mbox{}\\
\indent Wandb is a commercial product for monitoring the training of machine learning models. The monitoring includes:
\begin{itemize}
    \item collection of statistics and metrics,
    \item comparison of statistics,
    \item hardware monitoring,
    \item configuration monitoring,
    \item generation of training reports.
\end{itemize}

It can be synchronized with Tensorobard. The metrics logged on the Tensorboard can be viewed on the WanDB website and this option was utilized. Researchers in academia are given free resources to use. This opportunity was used; All experiments were recorded under the project \url{https://wandb.ai/deformer/ct-images}.

Thanks to the tool, it was possible to perform many experiments and describe them properly in this paper. Wandb was also used for online monitoring to check whether a model converges or not, since it would be wasteful to continue to train a model that has already stopped giving meaningful results.
\newpage
\subsection{Dataset}\mbox{}\\
\indent The dataset used to train the neural networks presented below consists of 28 pelvic scans. Each person scanned had diagnosed prostate cancer. All data have undergone an anonymization process.
\paragraph{Data format}\mbox{}\\
\indent The entire data set was stored in 28 NifTI files (.nii.gz) in the directory \url{/ravana/d3d_work/micorl/data/ct_images_prostate_32fixed} on the "dose3d" computer, located at the WFiIS Faculty in Kraków.

NIfTI (.nii or .nii.gz) is a file format for storing medical imaging data, especially in neuroimaging. The ".nii.gz" extension indicates a compressed version of the NIfTI file. Unlike DICOM, which stores each image slice separately, NIfTI stores the entire image volume in a single file with metadata. 

\paragraph{Data transformations}\mbox{}\\
\indent When loading a provisioned dataset for analysis, the CT scan images undergo multiple transformations to prepare them adequately. Initially, a central spatial crop is applied, focusing on the region of interest with dimensions (380,380,0). Next, each scan is resized to a resolution of 128 pixels in width, 128 pixels in height, and 32 layers (128x128x32) due to the graphic card memory constraints during models training. Then the intensity values are scaled to a range of [0, 1], based on specific window level and width parameters. 

As a result, a typical scan has a final shape of (1,128,128,32), corresponding to the (C,H,W,D) format. In certain scenarios, where some models were derived from video models, the scans are reshaped to the (1,32,128,128) format, which corresponds to (C,T,H,W), since a scan could be interpreted as a video.
\newpage
\subsection{Chosen approaches}
% \subsubsection{WGAN}
% \subsubsection{Autoencoder}
\subsubsection{VAE + LDM}
\input{}
The first attempt of generation of artificial scans was based on Variational Autoencoder architecture proposed in the \cite{rombach2022high}. Implementation was based on Monai\cite{Cardoso_MONAI_An_open-source_2022} example available on Github\footnote{\url{https://github.com/Project-MONAI/GenerativeModels/blob/e7cc989cdce440a7bff1cce22fff1caf760f39cd/tutorials/generative/3d_autoencoderkl/3d_autoencoderkl_tutorial.ipynb}} and adjusted to Pytorch Lightning. The final implementation is available in the project repository in file.

\paragraph{Model configuration}

\paragraph{Training}

\begin{figure}[H]
\centering
\begin{subfigure}[h]{.45\linewidth}
    \includegraphics[width=\linewidth]{detailed_engineering/Monai Autoencoder/charts/generator_loss.png}
    \caption{Caption}
    \label{fig:enter-label}
\end{subfigure}
\hfill
\begin{subfigure}[h]{.45\linewidth}
    \includegraphics[width=\linewidth]{detailed_engineering/Monai Autoencoder/charts/val_generator_loss.png}
    \caption{Caption}
    \label{fig:enter-label}
\end{subfigure}
\hfill
\begin{subfigure}[h]{.45\linewidth}
    \includegraphics[width=\linewidth]{detailed_engineering/Monai Autoencoder/charts/discriminator_loss.png}
    \caption{Caption}
    \label{fig:enter-label}
\end{subfigure}
\hfill
% \begin{subfigure}[h]{.45\linewidth}
%     \includegraphics[width=\linewidth]{detailed_engineering/Monai Autoencoder/charts/Section-4-Panel-4-d216pe2qa.png}
%     \caption{Caption}
%     \label{fig:enter-label}
% \end{subfigure}
% \hfill
% \begin{subfigure}[h]{.45\linewidth}
%     \includegraphics[width=\linewidth]{detailed_engineering/Monai Autoencoder/charts/Section-4-Panel-5-z2xepgyu7.png}
%     \caption{Caption}
%     \label{fig:enter-label}
% \end{subfigure}
%\begin{subfigure}[b]
%     \includegraphics[width=0.5\linewidth]{detailed_engineering/Monai Autoencoder/charts/Section-4-Panel-2-bm1y05a9m.png}
%     \caption{Caption}
%     \label{fig:enter-label}
% \end{subfigure}
% \begin{subfigure}[b]
%     \includegraphics[width=0.5\linewidth]{detailed_engineering/Monai Autoencoder/charts/Section-4-Panel-3-dkwhik6ki.png}
%     \caption{Caption}
%     \label{fig:enter-label}
% \end{subfigure}
% \begin{subfigure}[b]
%     \includegraphics[width=0.5\linewidth]{detailed_engineering/Monai Autoencoder/charts/Section-4-Panel-4-d216pe2qa.png}
%     \caption{Caption}
%     \label{fig:enter-label}
% \end{subfigure}
% \begin{subfigure}[b]
%     \includegraphics[width=0.5\linewidth]{detailed_engineering/Monai Autoencoder/charts/Section-4-Panel-5-z2xepgyu7.png}
%     \caption{Caption}
%     \label{fig:enter-label}
% \end{subfigure}
\end{figure}


\paragraph{Results}


\paragraph{DDPM model based on Monai Autoencoder}

\paragraph{Training}
\begin{figure}[H]
\centering
\begin{subfigure}[h]{.45\linewidth}
    \includegraphics[width=\linewidth]{detailed_engineering/Monai Diffusion - Attempt 1/charts/step_loss.png}
    \caption{Caption}
    \label{fig:enter-label}
\end{subfigure}
\hfill
\begin{subfigure}[h]{.45\linewidth}
    \includegraphics[width=\linewidth]{detailed_engineering/Monai Diffusion - Attempt 1/charts/val_loss.png}
    \caption{Caption}
    \label{fig:enter-label}
\end{subfigure}
\end{figure}

\begin{figure}[H]
\centering
\begin{subfigure}[h]{.45\linewidth}
    \includegraphics[width=\linewidth]{detailed_engineering/Monai Diffusion - Attempt 2/charts/step_loss.png}
    \caption{Caption}
    \label{fig:enter-label}
\end{subfigure}
\hfill
\begin{subfigure}[h]{.45\linewidth}
    \includegraphics[width=\linewidth]{detailed_engineering/Monai Diffusion - Attempt 2/charts/val_loss.png}
    \caption{Caption}
    \label{fig:enter-label}
\end{subfigure}
\end{figure}


% \hfill
% \begin{subfigure}[h]{.45\linewidth}
%     \includegraphics[width=\linewidth]{detailed_engineering/Monai Diffusion - Attempt 1/charts/}
%     \caption{Caption}
%     \label{fig:enter-label}
% \end{subfigure}
% \hfill
% \begin{subfigure}[h]{.45\linewidth}
%     \includegraphics[width=\linewidth]{detailed_engineering/Monai Diffusion - Attempt 1/charts/Section-4-Panel-3-dn8bpp6rt.png}
%     \caption{Caption}
%     \label{fig:enter-label}
% \end{subfigure}
% \hfill
% \begin{subfigure}[h]{.45\linewidth}
%     \includegraphics[width=\linewidth]{detailed_engineering/Monai Diffusion - Attempt 1/charts/Section-4-Panel-4-z2xepgyu7.png}
%     \caption{Caption}
%     \label{fig:enter-label}
% \end{subfigure}
% \begin{subfigure}[h]{.45\linewidth}
%     \includegraphics[width=\linewidth]{detailed_engineering/Monai Diffusion - Attempt 1/charts/Section-4-Panel-5-4t8ldmpb8.png}
%     \caption{Caption}
%     \label{fig:enter-label}
% \end{subfigure}
% \end{figure}

\paragraph{Model configuration}

\paragraph{Training}

\paragraph{Results}




\newpage
\subsubsection{Transformers CT: VQVAE + Transformers}
The VQVAE\footnote{\url{https://github.com/FirasGit/transformers_ct_reconstruction}} model and its complementation are based on the one presented in article\cite{khader_transformers_2023}.

Due to time constraints, the VQGAN and transformer models presented in the work have not been trained. Only VQVAE was trained and the result of this process is presented below. 


\paragraph{VQVAE}\mbox{}
\subparagraph{Configuration}\mbox{}\\

\begin{table}[h!]
\centering
\begin{tabular}{|l|l|}
\hline
\textbf{Parameter} & \textbf{Value} \\
\hline
\multicolumn{2}{|c|}{\textbf{Training}} \\
\hline
Accelerator & GPU \\
\hline
Devices & 2 \\
\hline
Precision & 32 \\
\hline
Strategy & DDP \\
\hline
Maximum Epochs & 10001 \\
\hline
\multicolumn{2}{|c|}{\textbf{Model}} \\
\hline
Input Channels & 1 \\
\hline
Output Channels & 1 \\
\hline
Embedding Channels & 8 \\
\hline
Number of Embeddings & 16384 \\
\hline
Spatial Dimensions & 3 \\
\hline
Hidden Channels & [32, 64, 128, 256] \\
\hline
Kernel Sizes & [3, 3, 3, 3] \\
\hline
Strides & [1, 2, 2, 2] \\
\hline
Embedding Loss Weight & 1 \\
\hline
Beta & 1 \\
\hline
Loss Function & L1 \\
\hline
Deep Supervision & 0 \\
\hline
Use Attention & [False, False, True, True] \\
\hline
Normalization & Group (num\_groups: 4, affine: True) \\
\hline
Sample Every N Epochs & 20 \\
\hline
Learning Rate & 5e-6 \\
\hline
\multicolumn{2}{|c|}{\textbf{Dataset}} \\
\hline
Caching & Disk \\
\hline
Path & /ravana/d3d\_work/micorl/data/ct\_images\_prostate\_32fixed/ \\
\hline
Image Size & 128 \\
\hline
Number of Slices & 32 \\
\hline
Window Width & 400 \\
\hline
Window Level & 60 \\
\hline
\end{tabular}
\caption{Configuration of the VQVAE approach.}
\label{table:training_params}
\end{table}

\subparagraph{Training}\mbox{}\\

The model was trained for 10000 epochs. As visible in the plots below, all losses started to slow down after epoch 2000. 

\begin{figure}[H]
\minipage{0.49\textwidth}
\includegraphics[width=\linewidth]{detailed_engineering/German VQVAE/charts/train_l1.png}
\caption{Training.}
\endminipage\hfill
\minipage{0.49\textwidth}
\includegraphics[width=\linewidth]{detailed_engineering/German VQVAE/charts/val_l1.png}
\caption{Validation.}
\endminipage
\caption{L1 loss between input and its reconstruction. Lower is better.}
\end{figure}

\begin{figure}[H]
\minipage{0.49\textwidth}
\includegraphics[width=\linewidth]{detailed_engineering/German VQVAE/charts/train_l2.png}
\caption{Training.}
\endminipage\hfill
\minipage{0.49\textwidth}
\includegraphics[width=\linewidth]{detailed_engineering/German VQVAE/charts/val_l2.png}
\caption{Validation.}
\endminipage
\caption{L2 loss between input and its reconstruction. Lower is better.}
\end{figure}

\begin{figure}[H]
\minipage{0.49\textwidth}
\includegraphics[width=\linewidth]{detailed_engineering/German VQVAE/charts/train_ssim.png}
\caption{Training.}
\endminipage\hfill
\minipage{0.49\textwidth}
\includegraphics[width=\linewidth]{detailed_engineering/German VQVAE/charts/val_ssim.png}
\caption{Validation.}
\endminipage
\caption{SSIM loss between input and its reconstruction. Higher is better.}
\end{figure}




\subparagraph{Results}\mbox{}\\

The best results were obtained for a checkpoint of the model from the 8499 epoch. At that point, model reconstructions seem almost indistinguishable from the original, as is shown in the figure \ref{fig:german_vqvae_best}.

\begin{figure}[H]
    \centering
    \includegraphics[width=\linewidth]{detailed_engineering/German VQVAE/charts/best_german_vqvae.png}
    \caption{The best quality reconstruction achieved. Epoch 8499, step 10540. Top - input, middle - reconstruction, bottom their difference}
    \label{fig:german_vqvae_best}
\end{figure}


\subsubsection{Medical Diffusion: VQGAN + LDM}
The Medical Diffusion\footnote{\url{https://github.com/FirasGit/medicaldiffusion}} model and its implementation are based on the work presented in the article \cite{https://doi.org/10.48550/arxiv.2211.03364}. 

\paragraph{Medical Diffusion: VQVAE}\mbox{}\\
\subparagraph{Configuration}\mbox{}\\

\begin{table}[H]
\centering
\begin{tabular}{|l|l|}
\hline
\textbf{Parameter} & \textbf{Value} \\
\hline
\multicolumn{2}{|c|}{\textbf{Training}} \\
\hline
Batch Size & 1 \\
\hline
Seed & 42 \\
\hline
Epochs & 15000 \\
\hline
Training Ratio & 0.9 \\
\hline
Number of Nodes & 1 \\
\hline
Gradient Clip Value & 1.0 \\
\hline
Metric to Monitor & val/recon\_loss \\
\hline
Metric Mode & Min \\
\hline
Device & CUDA \\
\hline
\multicolumn{2}{|c|}{\textbf{Model}} \\
\hline
Image Channels & 1 \\
\hline
Embedding Dimension & 8 \\
\hline
Number of Codes & 16384 \\
\hline
Number of Hidden Layers & 16 \\
\hline
Learning Rate & 5e-6 \\
\hline
Downsample Factors & [2, 2, 2] \\
\hline
Discriminator Channels & 64 \\
\hline
Discriminator Layers & 3 \\
\hline
Discriminator Iteration Start & 10000 \\
\hline
Discriminator Loss Type & Hinge \\
\hline
Image GAN Weight & 1.0 \\
\hline
Video GAN Weight & 1.0 \\
\hline
L1 Weight & 4.0 \\
\hline
GAN Feature Weight & 4.0 \\
\hline
Perceptual Weight & 4.0 \\
\hline
I3D Feature & False \\
\hline
Restart Threshold & 1.0 \\
\hline
No Random Restart & False \\
\hline
Normalization Type & Group \\
\hline
Padding Type & Replicate \\
\hline
Number of Groups & 32 \\
\hline
\multicolumn{2}{|c|}{\textbf{Dataset}} \\
\hline
Caching & Disk \\
\hline
Path & \url{/ravana/d3d\_work/micorl/data/ct\_images\_prostate\_32fixed/} \\
\hline
Image Size & 128 \\
\hline
Number of Slices & 32 \\
\hline
Window Width & 400 \\
\hline
Window Level & 60 \\
\hline
\end{tabular}
\caption{Configuration of the Medical Diffusion VQGAN.}
\label{table:meta_vqgan_params}
\end{table}

\subparagraph{Training}\mbox{}\\

The training was meant to last 15000 epochs, but it was manually stopped approximately at epoch 8000 due to the fact that for a long time model's reconstruction and perceptual loss did not change.

\begin{figure}[H]
\minipage{0.49\textwidth}
\includegraphics[width=\linewidth]{detailed_engineering/Meta VQGAN/charts/Section-2-Panel-13-1qhe42yar.png}
\caption{Reconstruction loss during the training.}
\endminipage\hfill
\minipage{0.49\textwidth}
\includegraphics[width=\linewidth]{detailed_engineering/Meta VQGAN/charts/Section-4-Panel-3-hkj1c12xb.png}
\caption{Reconstruction loss during the validation.}
\endminipage
\caption{Reconstruction loss during the training and the validation. Lower is better.}
\end{figure}

\begin{figure}[H]
\minipage{0.49\textwidth}
\includegraphics[width=\linewidth]{detailed_engineering/Meta VQGAN/charts/Section-2-Panel-7-jxu137h9u.png}
\caption{Perceptual loss during the training.}
\endminipage\hfill
\minipage{0.49\textwidth}
\includegraphics[width=\linewidth]{detailed_engineering/Meta VQGAN/charts/Section-4-Panel-0-9utd8c9z0.png}
\caption{Perceptual loss during the validation.}
\endminipage
\caption{Perceptual loss during the training and the validation. Lower is better.}
\end{figure}

\begin{figure}[H]
\minipage{0.49\textwidth}
\includegraphics[width=\linewidth]{detailed_engineering/Meta VQGAN/charts/Section-2-Panel-11-4ox8mpc2c.png}
\caption{Commitment loss during the training.}
\endminipage\hfill
\minipage{0.49\textwidth}
\includegraphics[width=\linewidth]{detailed_engineering/Meta VQGAN/charts/Section-4-Panel-2-2k8ixubhi.png}
\caption{Commitment loss during the validation.}
\endminipage
\caption{Commitment loss during the training and the validation.}
\end{figure}


\begin{figure}[H]
\minipage{0.49\textwidth}
\includegraphics[width=\linewidth]{detailed_engineering/Meta VQGAN/charts/Section-2-Panel-1-5nrgzgmoj.png}
\caption{Perplexity during the training.}
\endminipage\hfill
\minipage{0.49\textwidth}
\includegraphics[width=\linewidth]{detailed_engineering/Meta VQGAN/charts/Section-4-Panel-1-njiwngfn7.png}
\caption{Perplexity during the validation.}
\endminipage
\caption{Codebook utilisation during the training and the validation. Lower is better.}
\end{figure}

\subparagraph{Results}\mbox{}\\

The training was finished with low reconstruction and perceptual losses. The resulting reconstruction $\hat{x}$ seems almost indistinguishable from the original $x$, as shown in the figure \ref{fig:md-vqgan-comparison}.

\begin{figure}[H]
    \centering
    \includegraphics[width=\linewidth]{reports/meta_vqgan_reconstruction_comparison.png}
    \caption{Top - input, middle - reconstruction, bottom - difference.}
    \label{fig:md-vqgan-comparison}
\end{figure}



\paragraph{Medical Diffusion LDM}\mbox{}\\
\paragraph{Model configuration}

\paragraph{Model configuration}\mbox{}\\

\begin{table}[H]
\centering
\begin{tabular}{|l|l|}
\hline
\textbf{Parameter} & \textbf{Value} \\
\hline
\multicolumn{2}{|c|}{\textbf{Training}} \\
\hline
Batch Size & 5 \\
\hline
Seed & 42 \\
\hline
Epochs & 15000 \\
\hline
Training Ratio & 0.9 \\
\hline
Number of Nodes & 1 \\
\hline
Number of Devices & 1 \\
\hline
Device & CUDA \\
\hline
Float precision & 32 \\
\hline
Metric to Monitor & val/loss \\
\hline
Metric Mode & Min \\
\hline
\multicolumn{2}{|c|}{\textbf{Model}} \\
\hline
Learning Rate & 0.0001 \\
\hline
Loss Type & L1 \\
\hline
VQGAN Checkpoint Path & ./models/meta_vqgan/checkpoints/meta_vqgan-v5.ckpt \\
\hline
Probability of Focus Present & 0.0 \\
\hline
Gradient Scaler Enabled & False \\
\hline
Focus Present Mask & Null \\
\hline
Max Gradient Norm & Null \\
\hline
\multicolumn{2}{|c|}{\textbf{EMA Configuration}} \\
\hline
Decay & 0.995 \\
\hline
Start Step & 2000 \\
\hline
Update Every Step & 10 \\
\hline
\multicolumn{2}{|c|}{\textbf{Scan Parameters}} \\
\hline
Image Size & 64 \\
\hline
Depth Size & 16 \\
\hline
Number of Channels & 8 \\
\hline
\multicolumn{2}{|c|}{\textbf{Scheduler Parameters}} \\
\hline
Timesteps & 300 \\
\hline
\multicolumn{2}{|c|}{\textbf{UNet Parameters}} \\
\hline
Dimension Multipliers & [1, 2, 4, 8] \\
\hline
\multicolumn{2}{|c|}{\textbf{Dataset Configuration}} \\
\hline
Caching & Disk \\
\hline
Path & /ravana/d3d_work/micorl/data/ct_images_prostate_32fixed/ \\
\hline
Image Size & 128 \\
\hline
Number of Slices & 32 \\
\hline
Window Width & 400 \\
\hline
Window Level & 60 \\
\hline
\end{tabular}
\caption{Parameters for the Meta Diffuser}
\label{table:meta_diffuser_params}
\end{table}



\paragraph{Training}\mbox{}\\
\begin{figure}[H]
\minipage{0.49\textwidth}
\includegraphics[width=\linewidth]{detailed_engineering/Meta Diffusion/charts/train_loss.png}
\caption{Loss during the training. Lower is better.}
\endminipage\hfill
\minipage{0.49\textwidth}
\includegraphics[width=\linewidth]{detailed_engineering/Meta Diffusion/charts/val_loss.png}
\caption{Loss during the validation. Lower is better.}
\endminipage
\end{figure}

Additional metrics evaluating the quality of the generated scan.
\begin{figure}[H]
\minipage{0.49\textwidth}
\includegraphics[width=\linewidth]{detailed_engineering/Meta Diffusion/charts/train_l1_epoch.png}

\endminipage\hfill
\minipage{0.49\textwidth}
\includegraphics[width=\linewidth]{detailed_engineering/Meta Diffusion/charts/val_l1_epoch.png}

\endminipage
\caption{L1 between two generated CT scans and two random samples from training/validation datasets. Lower is better.}
\end{figure}

\begin{figure}[H]
\minipage{0.49\textwidth}
\includegraphics[width=\linewidth]{detailed_engineering/Meta Diffusion/charts/train_l2_epoch.png}

\endminipage\hfill
\minipage{0.49\textwidth}
\includegraphics[width=\linewidth]{detailed_engineering/Meta Diffusion/charts/val_l2_epoch.png}

\endminipage
\caption{L2 between two generated CT scasn and two random samples from training/validation datasets. Lower is better.}
\end{figure}

\begin{figure}[H]
\minipage{0.49\textwidth}
\includegraphics[width=\linewidth]{detailed_engineering/Meta Diffusion/charts/train_ssim_epoch.png}

\endminipage\hfill
\minipage{0.49\textwidth}
\includegraphics[width=\linewidth]{detailed_engineering/Meta Diffusion/charts/val_ssim_epoch.png}

\endminipage
\caption{MSSIM between two generated Ct scans and two random samples from training/validation datasets. Higher is better.}
\end{figure}

\paragraph{Results}\mbox{}\\

% \begin{figure}[H]
%     \centering
%     \includegraphics[width=\linewidth]{}
%     \caption{Caption}
%     \label{fig:enter-label}
% \end{figure}
\begin{figure}[H]
    \centering
    % \includegraphics{}
     \animategraphics[width=\textwidth, loop, autoplay]{5}%frame rate
    {detailed_engineering/Meta Diffusion/generation/layer-}%path to figures
    {0}%start index
    {31}%end index
    \caption{Random layer of synthetic CT scan generated from Gaussian noise.}
    \label{fig:my_label}
\end{figure}

\begin{figure}[H]
    \centering
    \includegraphics[width=\linewidth]{detailed_engineering/Meta Diffusion/charts/meta_diffusion_comparison.png}
    \caption{Top - first sample from trianing dataset, middle - synthetic CT scan, bottom - L1 difference between them.}
    \label{fig:ldm-success-comparison}
\end{figure}


\newpage
\subsection{Evaluation}

In the table below, autoencoders with their reconstruction quality measurements are presented. 
In addition, the Medical Diffusion $LDM$ generation quality was measured.

In order to evaluate the quality of the generated dataset, the FID score was calculated. Since the FID score is related to images, not scans or videos, each layer has been analyzed separately. 
A synthetic dataset of 28 images (length of the original data set) was created (Fig. \ref{fig:synthetic-dataset}). Then each corresponding layer of scans was analyzed by FID. The mean FID score for each layer in the synthetic dataset is equal to 77.076. 

The number is high compared to the ones in the VQGAN paper\cite{esser2021tamingtransformershighresolutionimage}. It could be caused by the fact that LDM may not generate the internals of the body in the correct order, as can be seen in the figure \ref{fig:ldm-success-comparison}. However, it should be acknowledged by a radiology specialist or doctor of medicine.
\begin{table}[h!]
\centering
\begin{tabular}{|c|c|c|c|}
\hline
\textbf{Metric} & VAE & VQVAE1 & Medical Diffusion VQVAE \\
\hline
$\overline{L1}$ & 0.0309 & 0.0212 & 0.0144 \\
\hline
$\overline{L2}$ & 0.0043 & 0.0023 & 0.0007 \\
\hline
$\overline{SSIM}$ & 0.8690 & 0.9399 & 0.9795 \\
\hline
$\overline{LPIPS}$ & 2.2485 & 1.3820 & 0.8898 \\
\hline
\end{tabular}
\caption{Mean metrics of each reconstructing model between inputs and produced reconstructions. The entire dataset was used (28 samples).}
\label{table:metrics}
\end{table}

\begin{table}[h!]
\centering
\begin{tabular}{|c|c|}
\hline
\textbf{Metric} & Medical Diffusion LDM \\
\hline
$\overline{L1}$ & 0.0158 \\
\hline
$\overline{L2}$ & 0.0007 \\
\hline
$\overline{SSIM}$ & 0.9560 \\
\hline
$\overline{LPIPS}$ & 0.9831 \\
\hline
\end{tabular}
\caption{Generative quality of the Medical Diffusion LDM. Statistics were calculated between mean input sample, $\overline{x}$ and mean synthetic sample,   $\overline{\hat{x}}$.}
\label{table:metrics-ldm}
\end{table}


\begin{figure}[H]
    \centering
    \includegraphics[width=\linewidth]{reports/mean_true_vs_synthetic_comparison.png}
    \caption{Top - mean CT scan from the input dataset, middle - mean CT scan from the synthetic dataset, bottom - difference.}
    \label{fig:mean_true_synth_diff}
\end{figure}


% \input{monai}