\documentclass[a4paper,12pt]{article}

%te paczki zapewniają język polski
\usepackage[utf8]{inputenc}
\usepackage[english]{babel}
%\usepackage[T1]{fontenc}
% \usepackage{polski}

\usepackage{indentfirst} %pierwsza linia jest z wcięciem (w ang nie, stąd potrzeba tej paczki)
\usepackage{perpage} %the perpage package
\MakePerPage{footnote} %restartuje numerowanie footnotes

%potrzebna do wszelkich matematycznych rzeczy (macierze, specjalne znaki zbiorów, itp.)
\usepackage{amsmath}
\usepackage[normalem]{ulem}
\usepackage{amsfonts}
\usepackage{adjustbox}
%\usepackage{framed} %pozwala na ramki

\usepackage{graphicx, animate} %grafika i obrazy
\usepackage{wrapfig} %do umiejscawiania grafiki
\usepackage{color} %kolorki
\usepackage{geometry} %jakies dodatkowe ustawienia
\usepackage{array} %pionowe centrowanie tekstu w tabelach -> P{}
\usepackage{tabularx} %dostowany tabular, ktory automatycznie łamie zbyt długie komórki
\usepackage{float}
\usepackage{xurl} %dlugie linki
\usepackage[sorting=none, maxnames=4]{biblatex} %bibliografia
\addbibresource{bibliography.bib}
\usepackage{subcaption} %podpisy podwykresow
\usepackage{titlesec}
%definicja wymiarów stron
\geometry{hmargin={2cm, 2cm}, height=10.0in}

\usepackage{hyperref} %spis treści
\newcolumntype{P}[1]{>{\centering\arraybackslash}p{#1}}
\newcommand{\mmathbf}[1]{$\mathbf{#1}$}
\renewcommand{\thefootnote}{\fnsymbol{footnote}}

\newenvironment{conditions*}
  {
  \noindent gdzie:
  \par
  \vspace{\abovedisplayskip}\noindent
   \tabularx{\columnwidth}{>{$}l<{$} @{${}-{}$} >{\raggedright\arraybackslash}X}}
  {\endtabularx\par\vspace{\belowdisplayskip}}
  
\usepackage[linesnumbered]{algorithm2e} 
\usepackage{tikz-cd}
\usepackage{tikz}
\usetikzlibrary{positioning, shapes, shapes.geometric, arrows.meta, calc}


% K0D

% \usepackage[skip=1pt]{caption}
% \captionsetup[subfigure]{aboveskip=0pt}
\usepackage{listings ,pmboxdraw}
\lstset{
  basicstyle=\ttfamily,
  columns=fullflexible,
  keepspaces,
  literate=
  {┐}{\textSFiii}1%
  {└}{\textSFii}1%
  {┴}{\textSFvii}1%
  {┬}{\textSFvi}1%
  {├}{\textSFviii}{1}%
  {─}{\textSFx}1%
  {│}{\textSFxi}1%
  {┼}{\textSFv}1,
}

\usepackage{xcolor}
\usepackage{svg}

\definecolor{codegreen}{rgb}{0,0.6,0}
\definecolor{codegray}{rgb}{0.5,0.5,0.5}
\definecolor{codepurple}{rgb}{0.58,0,0.82}
\definecolor{backcolour}{rgb}{0.95,0.95,0.92}

\lstdefinestyle{mystyle}{
    backgroundcolor=\color{backcolour},   
    commentstyle=\color{codegreen},
    keywordstyle=\color{magenta},
    numberstyle=\tiny\color{codegray},
    stringstyle=\color{codepurple},
    basicstyle=\ttfamily\footnotesize,
    breakatwhitespace=false,         
    breaklines=true,                 
    captionpos=b,                    
    keepspaces=true,                 
    numbers=left,                    
    numbersep=5pt,                  
    showspaces=false,                
    showstringspaces=false,
    showtabs=false,                  
    tabsize=2
}

\lstset{style=mystyle}

% my parg
\newcommand{\myparagraph}[1]{\paragraph{#1}\mbox{}\\}

\usepackage{amsmath, amssymb, latexsym}
\usepackage{sidecap}
 
\usepackage{tikz}
\usetikzlibrary{decorations.pathreplacing}
 \usepackage[mode=build]{standalone}
\usepackage{import}


\let\oldref\ref
\renewcommand{\ref}[1]{(\oldref{#1})} % () przy referencjach

\newcommand{\myref}[1]{(\ref{#1})} % () przy referencjach