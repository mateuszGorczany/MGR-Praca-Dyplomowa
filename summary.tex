This project assessed the performance of the VAE, VQVAE, and VQGAN models for generating synthetic prostate CT scans. The implementation utilized modern GPUs and multi-GPU training methods. Tensorboard and WanDB were employed to monitor the AI model training. The execution environment was set up with cutting-edge technologies, Devbox and Rye, to ease future work. Furthermore, the project was initiated with a template that is widely adopted as a standard in the Data Science community.

 The results obtained allowed us to compare them and select the best one. This turned out to be the Medical Diffusion model (VQGAN + U-Net3D LDM) model\cite{khader2023medicaldiffusiondenoisingdiffusion}, which produces visually coherent and good quality prostate scans from Gaussian noise.  

\paragraph{The next steps}\mbox{}\\
\indent The next action that can be taken involves improving both the resolution and the dimensions of the scans. The enhancement in quality can be done by fine-tuning the parameters of the VQGAN model, with particular attention to the Autoencoder segment. A plausible method to accomplish this is to alter the codebook size and correspondingly adjust the scan size within the latent space.

Higher-quality images could possibly be also obtained using a transformer architecture that learns the connections between the encoding vectors and then generates images by the next embeeding prediction\cite{esser2021tamingtransformershighresolutionimage}\cite{khader_transformers_2023}. An another approach could be replacement of the denoising U-Net with the Diffusion Transformer\cite{peebles2023scalablediffusionmodelstransformers} or the Multi-Modal Diffusion Transformer\cite{esser2024scaling}, which is used in Stable Diffusion 3\cite{esser2024scaling}.  

Due to the fact that scans exceeding dimensions of 128x128 could not be handled by the graphics card, this particular size was employed for the majority of the models. In circumstances where it is necessary to produce larger scans, one can utilize the Fully Sharded Data Parallel technique or use a high-resolution technique mentioned in the VQGAN paper\cite{esser2021tamingtransformershighresolutionimage}. 

Nevertheless, in situations where the WFiIS "dose3d" computer operates at an insufficient speed or out-of-memory issues occur again, it is recommended to switch to a different computational system, such as an Athena supercomputer at Cyfronet AGH.

