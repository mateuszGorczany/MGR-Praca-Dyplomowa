\begin{figure}[H]
    \centering
    \begin{tikzpicture}[
    box/.style={draw, minimum width=2cm, minimum height=3cm, text width=1.8cm, align=center},
    smallbox/.style={draw, minimum width=1.5cm, minimum height=1.2cm, text width=1.3cm, align=center},
    arrow/.style={->, >=latex, thick},
    label/.style={font=\small}
]

% Components
\node[circle, draw] (input) {Input $x$};
\node[box, right=1cm of input] (encoder) {Encoder};

% Latent space
\node[right=2cm of encoder] (latent) {};
\node[circle, draw, above=0.3cm of latent] (mu) {$\mu$};
\node[circle, draw, below=0.3cm of latent] (sigma) {$\sigma$};
\node[circle, draw, minimum size=1cm, right=1cm of latent] (z) {$z$};

\node[box, right=1cm of z] (decoder) {Decoder};
\node[circle, draw, right=1cm of decoder] (output) {Output $\hat{x}$};

% Reparameterization trick
\node[circle, draw, fill=white, minimum size=0.5cm, above=0.5cm of z] (epsilon) {$\epsilon$};

% Connections
\draw[arrow] (input) -- (encoder);
\draw[arrow] (encoder) -- (mu);
\draw[arrow] (encoder) -- (sigma);
\draw[arrow] (mu) -- (z);
\draw[arrow] (sigma) -- (z);
\draw[arrow] (epsilon) -- (z);
\draw[arrow] (z) -- (decoder);
\draw[arrow] (decoder) -- (output);

% Labels
\node[label, below=0.2cm of input] {High-dimensional};
% \node[label, above=0.2cm of mu] {Latent space};
\node[label, below=0.2cm of sigma] {Latent space};
\node[label, above=0.2cm of epsilon] {$\sim \mathcal{N}(0,1)$};
\node[label, below=0.2cm of output] {Reconstructed};

% % KL Divergence
% \draw[<->, >=latex, bend left=30] ($(encoder.north east)+(0.2,0.2)$) to node[above, font=\small] {KL Divergence} ($(mu.north west)+(-0.2,0.2)$);

% Reconstruction Loss
\draw[<->, >=latex, bend right=30] ($(input.south west)+(1,-0.3)$) to node[below, font=\small] {Reconstruction Loss} ($(output.south east)+(-1.3,-0.2)$);

\end{tikzpicture}
    \caption{Visualization of Variational Autoencoder. Circles are tensors, rectangles are neural networks.}
    \label{fig:vae}
\end{figure}

\indent The main issue with autoencoders in the context of synthetic data generation is that we do not know what values should be assigned to the layer $z$ and passed through decoder in order to obtain meaningful results. The Variational Autoencoder addresses this problem.
This model extends the standard autoencoder by linking it with probability concepts: mean and standard deviation. VAE learns not only to create a compressed representation of the data, but also a distribution over that representation. This makes VAEs capable of generating new data samples that are similar to those in the training set. Thus, VAE is a suitable choice for image generation, data augmentation, and even anomaly detection.

\begin{figure}[H]
    \centering
    
\begin{tikzpicture}[
    neuron/.style={circle, draw, minimum size=0.5cm},
    layer/.style={rectangle, draw, rounded corners, minimum height=3cm, minimum width=1cm, fill opacity=0.2},
    arrow/.style={->, >=latex, thick},
    label/.style={font=\small}
]

% Color definitions
\colorlet{input}{gray!20}
\colorlet{encoder}{blue!40}
\colorlet{mu}{orange!20}
\colorlet{sigma}{blue!20}
\colorlet{latent}{green!20}
\colorlet{decoder}{red!20}

% Input layer
\node[layer, fill=input] (input) {};
\foreach \y in {1,...,4} {
    \node[neuron, fill=input] (i\y) at (input.west |- input.north) [yshift=-\y*0.60cm, xshift=0.5cm] {};
}

% Encoder layer
\node[layer, fill=encoder, right=1.5cm of input] (encoder) {};
\foreach \y in {1,...,3} {
    \node[neuron, fill=encoder] (e\y) at (encoder.west |- encoder.north) [yshift=-\y*0.8cm, xshift=0.5cm] {};
}

% Mu layer
\node[layer, fill=mu, right=1.5cm of encoder, minimum height=1.5cm, yshift=0.75cm] (mu) {};
\foreach \y in {1,...,2} {
    \node[neuron, fill=mu] (mu\y) at (mu.west |- mu.north) [yshift=-\y*0.5cm, xshift=0.5cm] {};
}

% Sigma layer
\node[layer, fill=sigma, below=0.5cm of mu, minimum height=1.5cm] (sigma) {};
\foreach \y in {1,...,2} {
    \node[neuron, fill=sigma] (sigma\y) at (sigma.west |- sigma.north) [yshift=-\y*0.5cm, xshift=0.5cm] {};
}

% Latent layer
\node[layer, fill=latent, right=1.5cm of mu, yshift=-0.75cm, minimum height=2cm] (latent) {};
\foreach \y in {1,...,2} {
    \node[neuron, fill=latent] (l\y) at (latent.west |- latent.north) [yshift=-\y*0.7cm, xshift=0.5cm] {};
}

% Decoder layer
\node[layer, fill=decoder, right=1.5cm of latent] (decoder) {};
\foreach \y in {1,...,3} {
    \node[neuron, fill=decoder] (d\y) at (decoder.west |- decoder.north) [yshift=-\y*0.7cm, xshift=0.5cm] {};
}

% Output layer
\node[layer, fill=input, right=1.5cm of decoder] (output) {};
\foreach \y in {1,...,4} {
    \node[neuron, fill=input] (o\y) at (output.west |- output.north) [yshift=-\y*0.65cm, xshift=0.5cm] {};
}

% Connections
\foreach \i in {1,...,4} {
    \foreach \j in {1,...,3} {
        \draw[blue!50, thick] (i\i) -- (e\j);
    }
}

\foreach \i in {1,...,3} {
    \foreach \j in {1,...,2} {
        \draw[orange!50, thick] (e\i) -- (mu\j);
        \draw[blue!50, thick] (e\i) -- (sigma\j);
    }
}

\foreach \i in {1,...,2} {
    \draw[green!50, thick] (mu\i) -- (l\i);
    \draw[green!50, thick] (sigma\i) -- (l\i);
}

\foreach \i in {1,...,2} {
    \foreach \j in {1,...,3} {
        \draw[red!50, thick] (l\i) -- (d\j);
    }
}

\foreach \i in {1,...,3} {
    \foreach \j in {1,...,4} {
        \draw[red!50, thick] (d\i) -- (o\j);
    }
}

% Labels
\node[label, above=0.2cm of input] {$x$};
\node[label, above=0.2cm of encoder] {$E$};
\node[label, above=0.2cm of mu] {$\mu(x)$};
\node[label, below=0.2cm of sigma] {$\sigma(x)$};
\node[label, above=0.2cm of latent] {$z \sim \mathcal{N}(\mu, \sigma)$};
\node[label, above=0.2cm of decoder] {$D$};
\node[label, above=0.2cm of output] {$\hat{x} = D(z)$};

\end{tikzpicture}
    \caption{Visualization of a simple VAE.}
    \label{fig:vae-simple}
\end{figure}

The figure[\ref{fig:vae-simple}] shows detailed visualization of a simple Variational Autoencoder. It is worth noticing that the encoder $E$ has two outputs, $\mu$ and $\sigma$, which produce $z$ using the reparameterization trick\footnote{\url{https://www.baeldung.com/cs/vae-reparameterization}}.

\begin{equation}
    z = \mu + \sigma\cdot\epsilon,
\end{equation}

where $\epsilon\sim\mathcal{N}(0,1)$.

\paragraph{Loss Function:}\mbox{}\\
The VAE loss is a combination of:
\begin{enumerate}
\item \textbf{Reconstruction Loss}, which ensures how well the reconstructed image matches the original input image, for example $L1$, $L2$, $MSE$,
\item \textbf{KL Divergence Loss} that ensures the latent space follows a Gaussian distribution.
\end{enumerate}
The loss formula:

\begin{equation}
    L = L_{reconstruction} + \mathbb{KL}(q(z | x) || p(z)),
\end{equation}

where 
\begin{itemize}
    \item $\mathbb{KL}$ - Kullback-Leibler divergence\footnote{\url{https://en.wikipedia.org/wiki/Kullback\%E2\%80\%93Leibler_divergence}}, measures how much two probability distributions differ from each other,
    \item $q(z | x) = \mathcal{N}(\mu, \sigma)$ - probability distribution of the latent representation $z$, in other words probability distribution of endocers output,
    \item $p(z) = \mathcal{N}(0,1)$ - expected probability distribution.
\end{itemize}

After substituting probabilities, the loss $L$ becomes\footnote{Broader explanation and Pytorch toy example can be found here \url{https://avandekleut.github.io/vae/}.}
\begin{equation}
    L = L_{reconstruction} + \mathbb{KL}(\mathcal{N}(\mu,\sigma), \mathcal{N}(0,1)).
\end{equation}

The difference between two Gaussians can be calculated using the identity\cite{7449}, 

\begin{equation}
    \mathbb{KL}\left( \mathcal{N}(\mu, \sigma) \parallel \mathcal{N}(0, 1) \right) = \sum_{x \in X} \left( \sigma^2 + \mu^2 - \log \sigma - \frac{1}{2} \right).
\end{equation}


% Train the VAE on prostate CT scans to learn an efficient latent space representation.
\paragraph{Generation of CT scans}\mbox{}\\

To generate new CT scans, we could sample the Gaussian noise tensor of shape $z$, which follows the normal distribution, and then pass it through the decoder.
If the model is properly trained, the decoder should reconstruct this latent representation into a realistic synthetic CT image. 

However, it is not guaranteed that the learned distribution $q(z)$ is always similar to $\mathbf{N}(0,1)$, especially when the $\mathbb{KL}$ loss is far from $0$. To mitigate this issue, aggregated posterior probability could be used. For example, by passing the entire dataset through the encoder and averaging resulted in $\mu$s to $\overline{\mu}$ and $\sigma$s to $\overline{\sigma}$ and then sampling from $\mathcal{N}(\overline{\mu}, \overline{\sigma})$. The alternative solution is to train another neural network that would learn the distribution of the latent representation of $z$ - $q(z)$. Such a neural network could be a Latent Diffusion Model described in the previous section.

% ### Key Differences from Traditional Autoencoders

% ### VAE Structure

% 1. **Encoder (Inference Network)**: The encoder maps an input \( x \) to a latent representation, but instead of outputting a single value, it outputs parameters of a probability distribution, typically a Gaussian distribution with a mean \( \mu \) and a variance \( \sigma^2 \). This is achieved by splitting the encoder's output into two parts:

% \[
% q(z|x) = \mathcal{N}(z; \mu(x), \sigma^2(x))
% \]

% where:
% - \( \mu(x) \) is the mean of the latent distribution for input \( x \),
% - \( \sigma^2(x) \) is the variance of the latent distribution for input \( x \),
% - \( z \) is the latent variable sampled from the distribution \( q(z|x) \).

% 2. **Reparameterization Trick**: To allow backpropagation through the stochastic sampling process, VAEs use the reparameterization trick. Instead of sampling \( z \) directly from \( \mathcal{N}(\mu(x), \sigma^2(x)) \), a random variable \( \epsilon \sim \mathcal{N}(0, 1) \) is sampled, and \( z \) is computed as:

% \[
% z = \mu(x) + \sigma(x) \cdot \epsilon
% \]

% This trick makes the sampling process differentiable, enabling gradient-based optimization.

% 3. **Decoder (Generative Network)**: The decoder reconstructs the input data from the latent variable \( z \). The decoder maps \( z \) back to the data space, producing a reconstruction \( \hat{x} \). Similar to traditional autoencoders, this is done through a neural network.

% ### VAE Loss Function

% The loss function in a VAE consists of two terms:

% 1. **Reconstruction Loss**: This measures how well the decoder can reconstruct the input from the latent variable \( z \). It is typically the Mean Squared Error (MSE) or binary cross-entropy, depending on the nature of the data:

% \[
% L_{reconstruction} = \mathbb{E}_{q(z|x)} [||x - \hat{x}||^2]
% \]

% 2. **KL Divergence (Regularization Term)**: The second term ensures that the learned latent distribution \( q(z|x) \) is close to a chosen prior distribution \( p(z) \) (usually a standard normal distribution \( \mathcal{N}(0, 1) \)). This regularization term is the Kullback-Leibler (KL) divergence between the approximate posterior \( q(z|x) \) and the prior \( p(z) \):

% \[
% D_{KL}(q(z|x) || p(z)) = \frac{1}{2} \sum_{i=1}^n (1 + \log(\sigma_i^2) - \mu_i^2 - \sigma_i^2)
% \]

% The total VAE loss is a combination of the reconstruction loss and the KL divergence:

% \[
% L_{VAE} = L_{reconstruction} + D_{KL}(q(z|x) || p(z))
% \]

% This loss encourages the model to reconstruct the input data accurately while ensuring that the latent variables follow the prior distribution.

% ### Applications

% 1. **Data Generation**: By sampling latent vectors from the learned distribution, VAEs can generate new data points similar to those in the training set.
   
% 2. **Anomaly Detection**: VAEs can detect anomalies by measuring the reconstruction error for unseen data; higher reconstruction errors often indicate anomalies.
   
% 3. **Data Augmentation**: VAEs can be used to generate additional synthetic data for training, improving model generalization.

% 4. **Latent Space Exploration**: The continuous and smooth latent space learned by VAEs allows for meaningful interpolations between data points.

% ---

% **References**:
% 1. Kingma, D. P., & Welling, M. (2013). *Auto-Encoding Variational Bayes*. arXiv preprint arXiv:1312.6114.
% 2. Doersch, C. (2016). *Tutorial on Variational Autoencoders*. arXiv preprint arXiv:1606.05908.
% \begin{tikzpicture}[
    box/.style={draw, minimum width=2cm, minimum height=3cm, text width=1.8cm, align=center},
    smallbox/.style={draw, minimum width=1.5cm, minimum height=1.2cm, text width=1.3cm, align=center},
    arrow/.style={->, >=latex, thick},
    label/.style={font=\small}
]

% Components
\node[circle, draw] (input) {Input $x$};
\node[box, right=1cm of input] (encoder) {Encoder};

% Latent space
\node[right=2cm of encoder] (latent) {};
\node[circle, draw, above=0.3cm of latent] (mu) {$\mu$};
\node[circle, draw, below=0.3cm of latent] (sigma) {$\sigma$};
\node[circle, draw, minimum size=1cm, right=1cm of latent] (z) {$z$};

\node[box, right=1cm of z] (decoder) {Decoder};
\node[circle, draw, right=1cm of decoder] (output) {Output $\hat{x}$};

% Reparameterization trick
\node[circle, draw, fill=white, minimum size=0.5cm, above=0.5cm of z] (epsilon) {$\epsilon$};

% Connections
\draw[arrow] (input) -- (encoder);
\draw[arrow] (encoder) -- (mu);
\draw[arrow] (encoder) -- (sigma);
\draw[arrow] (mu) -- (z);
\draw[arrow] (sigma) -- (z);
\draw[arrow] (epsilon) -- (z);
\draw[arrow] (z) -- (decoder);
\draw[arrow] (decoder) -- (output);

% Labels
\node[label, below=0.2cm of input] {High-dimensional};
% \node[label, above=0.2cm of mu] {Latent space};
\node[label, below=0.2cm of sigma] {Latent space};
\node[label, above=0.2cm of epsilon] {$\sim \mathcal{N}(0,1)$};
\node[label, below=0.2cm of output] {Reconstructed};

% % KL Divergence
% \draw[<->, >=latex, bend left=30] ($(encoder.north east)+(0.2,0.2)$) to node[above, font=\small] {KL Divergence} ($(mu.north west)+(-0.2,0.2)$);

% Reconstruction Loss
\draw[<->, >=latex, bend right=30] ($(input.south west)+(1,-0.3)$) to node[below, font=\small] {Reconstruction Loss} ($(output.south east)+(-1.3,-0.2)$);

\end{tikzpicture}
% Reparametrization trick


% \begin{tikzpicture}[
    node distance = 1cm and 2cm,
    every node/.style = {draw, minimum size=0.8cm},
    % square/.style = {sides=4},
    square/.style = {regular polygon,regular polygon sides=4},
    arr/.style = {->, >=stealth}
]

% Original graph
\node[square] (f1) at (0,0) {$f$};
\node[circle] (z1) at (0,-1.5) {$z$};
\node[square] (mu1) at (-1,-3) {$\mu$};
\node[square] (sigma1) at (1,-3) {$\sigma$};

\draw[arr] (mu1) -- (z1);
\draw[arr] (sigma1) -- (z1);
\draw[arr] (z1) -- (f1);

\node[below=0.5cm of sigma1, draw=none] {Original};

% Reparametrized graph
\node[square] (f2) at (5,0) {$f$};
\node[square] (z2) at (5,-1.5) {$z$};
\node[square] (mu2) at (4,-3) {$\mu$};
\node[square] (sigma2) at (5,-3) {$\sigma$};
\node[circle] (epsilon) at (6,-3) {$\varepsilon$};

\draw[arr] (mu2) -- (z2);
\draw[arr] (sigma2) -- (z2);
\draw[arr] (epsilon) -- (z2);
\draw[arr] (z2) -- (f2);

\node[below=0.5cm of sigma2, draw=none] {Reparametrized};

\end{tikzpicture}
